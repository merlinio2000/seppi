\section{Linearisierung}

\subsubsection{Repetition Tangentengleichung}

Dient als Annäherung für $f(x)$ in der Nähe von $x_0$ (Linearisierung): \\
$g(x) = f(x_0) + f'(x_0)(x - x_0)$




\subsection{Jacobi-Matrix}
Sozusagen wie Tangentengleichung aber für mehrere Variablen
$$\vec{y} = \vec{f}(\vec{x}) =
	\begin{pmatrix}
		y_1 = f_1(x_1, ..., x_n) \\
		\hdots                   \\
		y_m = f_m(x_1, ..., x_n) \\
	\end{pmatrix}
$$

Die Jacobi-Matrix enthält sämtliche Partielle Ableitungen 1. Ordnung von $\vec{f}$. \\
Auf jeder Spalte bleibt die funktion $f_j$ die gleiche und in den
Zeilen $x_i$ $\rightarrow$ $\frac{\partial f_j}{\partial x_i}$

\scalemath{1.4}{
	\vec{Df}(x) =
	\begin{pmatrix}
		\frac{\partial f_1}{\partial x_1}(x) & \frac{\partial f_1}{\partial x_2}(x) & \vdots & \frac{\partial f_1}{\partial x_n} \\
		\frac{\partial f_2}{\partial x_1}(x) & \frac{\partial f_2}{\partial x_2}(x) & \vdots & \frac{\partial f_2}{\partial x_n} \\
		\hdots                               & \hdots                               & \ddots & \hdots                            \\
		\frac{\partial f_m}{\partial x_1}(x) & \frac{\partial f_m}{\partial x_2}(x) & \vdots & \frac{\partial f_m}{\partial x_n} \\
	\end{pmatrix}
}












